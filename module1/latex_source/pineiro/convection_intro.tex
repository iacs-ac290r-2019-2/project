Thermal convection is a natural process that drives fluids in motion. This phenomenon occurs when thermal energy is transferred from one location to another by the movement of fluids. There exits two general kinds of thermal convection: natural convection and forced convection. The former occurs when fluids are driven in motion by buoyancy forces, which are affected by changes in density caused by the surrounding temperature. The particles near a hot surface become excited, thus scattering and causing the fluid to be less dense, while the particles near the cold surface sink. Therefore, causing the hotter volume to transfer heat towards the cooler volume of that fluid. This interaction between the varying densities leads to various fluid dynamics, which we observe in our simulation. On the other hand, there exists forced convection, which occurs when a fluid is forced to flow over the surface by an internal source such as fans and pumps, creating an artificially induced convection current. This report will focus on natural convection. 

Thermal convection has many applications, and understanding its dynamics is vital in understanding the the nature of the universe we live in. Researchers have been trying to understand the mechanics of this naturally occurring phenomenon for quite some time. Henri Claude Bénard in 1900 observed a fluid behaving in an atypical fashion on a Petri dish placed on a hot plate. Although, Bénard's setup was a result from surface tension effects, it was only years later that a British mathematician, Lord Rayleigh successfully determined under what conditions this thermal convection occurred. 

Since then, because of the prevalence of thermal convection, it has been at the center of many research studies. In 1978, an article was published in the Journal of Agricultural Engineering Research by JM Bruce, shedding light into cattle ventilation [1]. By studying the various geometries and relevant to buildings, Bruce developed a theory which lead to design procedures for open-ridge and slotted-roof systems of ventilation for cattle buildings. This example highlights the importance of understanding thermal convection and its impact in our everyday lives. 

