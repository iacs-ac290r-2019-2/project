\subsection{Spatial Discretization}
The spatial part of the PDE problem is handled with the Finite Element Method (FEM). The essential idea of FEM is to first derive the weak form of the original problem, then represent the weak form using a set of functional basis defined on the discretized space elements, and in the end solve the problem under the new representation. 

Here we take the example provided by T. J. R. Hughs in his \emph{The Finite Element Method}, sect 1.12 - 1.15. We start with an 1D Poisson equation with boundary conditions:
\begin{align}
    -\partial_x^2 u=f(x),&\quad x\in(0,1), \\
    -\partial_x u(0)=\mathfrak{h},&\quad u(1)=\mathfrak{g}.
\end{align}
The weak form says to find $u\in\mathcal{S}$, \emph{s.t.} $\forall w\in\mathcal{V}$,
\begin{align}
    \int_0^1\!\partial_x w\partial_x u~\mathrm{d}x
    =\int_0^1\! w f~\mathrm{d}x + w(0)\mathfrak{h}.
\end{align}
We discretize the $x$ domain by $n$ elements \begin{align*}
    \begin{gathered}
        (x_1,x_2),~(x_2,x_3),~\ldots,~(x_n, x_{n+1})\\
        \mathrm{where}~x_1=0,~x_{n+1}=1
    \end{gathered}
\end{align*}
and choose the functional basis to be
\begin{align}
    N_i(x)=\left\{\begin{aligned}
        &\frac{x-x_{i-1}}{x_i-x_{i-1}}&,&\quad x\in(x_{i-1},x_i]\\
        &\frac{x_{i+1}-x}{x_{i+1}-x_i}&,&\quad x\in(x_{i},x_{i+1}]\\
        &0&,&\quad \mathrm{elsewhere}
    \end{aligned}\right.
    ,\quad i=1,2,\ldots,n+1. \\
\end{align}
The weak form can now be written in the discretized form, \emph{i.e.} the Galerkin statement
\begin{align}
    \int_0^1\!\partial_x w^h\partial_x u^h~\mathrm{d}x
    =\int_0^1\! w^h f~\mathrm{d}x + w^h(0)\mathfrak{h},
\end{align}
where the functions should be represented by
\begin{align}
    u^h=\sum_{i=1}^nu_iN_i+\mathfrak{g}N_{n+1},\quad
    w^h=\sum_{i=1}^nw_iN_i.
\end{align}
This gives us the equation for the coefficients $u_i$ and $w_i$
\begin{align}
    \sum_{i=1}^n w_i \left(\sum_{j=1}^n K_{ij} u_{j} - F_i\right)=0,\label{eqn:weak_dis}
\end{align}
where
\begin{align}
    K_{ij}=\int_0^1\!\partial_xN_i\partial_xN_j~\mathrm{d}x,\quad
    F_i=N_i(0)\mathfrak{h}+\int_0^1\!N_if~\mathrm{d}x-\mathfrak{g}\int_0^1\!\partial_xN_i\partial_xN_{n+1}~\mathrm{d}x
\end{align}
are known to us. Note that the weak form should work for any $w$, thus from (\ref{eqn:weak_dis}) we must have
\begin{align}
    \sum_{j=1}^n K_{ij} u_{j} - F_i=0, 
\end{align}
which is a linear equation system whose solution gives us the $u$ we want. 


\subsection{Temporal Discretization}
The temporal part of the PDE is handled by the SDIRK method, a member of the implicit Runge-Kutta family. The implicity helps to keep the numerical solution stable. 

\subsection{Non-linear Solver}
The S-N equation is non-linear. The solver responsible for this part is line-search based, which means a direction for solution update will first be determined and the length of the step moving towards that direction will be then estimated. Here the determination of the direction is achieved using the Newton method. 