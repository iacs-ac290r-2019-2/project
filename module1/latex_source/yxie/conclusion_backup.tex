In this work, we studied the Rayleigh-Bénard convection of $Ra=10^{10}$, $Pr=1$ through numerical simulation on a high performance computing platform. From the simulation results, we provided a thorough analysis of the temperature distribution over time. We observed the formation and mixing of plumes in RBC, as well as natural eddies formed by changes in density. The time dependent temperature profile provided a more intuitive description of the mixing process. In addition, the Nusselt number of the system was calculated as a function of time. Towards the thermal equilibrium, the Nusselt number converged to approximately 120. A value often described as identifying what some researchers describe as turbulent flow. 

Due to limited computational resources, the simulation performed in this work did not arrive at a physical time which could be considered long enough. That is, although much occurred within the observed timeframe, much is left to be said about the dynamic flow which will occur after. For the future work, we envision running the simulation on more CPUs with longer a longer timeframe. Another interesting work would be to model thermal within other objects and environments, a common research topic in various fields.